\newcommand \rsword [1]{\textit{\word{#1}}}

\section{第一题}

\begin{tabular}{|l|l|} \hline
\rsword{nohobe} & 我在打他 \\ \hline
\rsword{kahalune} & 我们将打你 \\ \hline
\rsword{nokoho’ibe} & 我俩在打你 \\ \hline
\rsword{nolenufu’inagihe} & 因为我俩在刺你们 \\ \hline
\rsword{nolifi’ibe} & 你俩在刺我们 \\ \hline
\rsword{nofunagihe} & 因为我在刺他 \\ \hline
\rsword{nofine} & 你在刺他 \\ \hline
\rsword{nifila’ibe} & 你俩将刺我 \\ \hline
\rsword{nonahatagihe} & 因为你在打我 \\ \hline
\rsword{lenahalube} & 我将打你们 \\ \hline
\rsword{nahalanagihe} & 因为你们将打我 \\ \hline
\rsword{lahala’ibe} & 你俩将打我们 \\ \hline
\rsword{nofutagihe} & 因为我们在刺他 \\ \hline
\rsword{lenifilu’ibe} & 我俩将刺你们 \\ \hline
\rsword{noho’inagihe} & 因为我俩在打他 \\ \hline
\end{tabular}

这些是贝纳贝纳语的一些动词形式及其汉语翻译。贝纳贝纳语属于跨新几内亚语系,还剩四万五千人在用。

这道题的作者是戴谊凡先生。这是一道很温和的题目,平均得分也很高,但刚开始做它的时候,
我卡了很久也毫无进展。下面我将重现我的推理过程。

\subsection{推理过程}

简单的观察就可以发现:

\begin{tabular}{|l|l|} \hline
\rsword{nohobe} & 我在打他 \\ \hline
\rsword{kahalune} & 我们将打你 \\ \hline
\rsword{nokoho’ibe} & 我俩在打你 \\ \hline
\rsword{nolifi’ibe} & 你俩在刺我们 \\ \hline
\rsword{nofine} & 你在刺他 \\ \hline
\rsword{nifila’ibe} & 你俩将刺我 \\ \hline
\rsword{lenahalube} & 我将打你们 \\ \hline
\rsword{lahala’ibe} & 你俩将打我们 \\ \hline
\rsword{lenifilu’ibe} & 我俩将刺你们 \\ \hline
\end{tabular}
\quad
\begin{tabular}{|l|l|} \hline
\rsword{nolenufu’in\highlight{agihe}} & \highlight{因为}我俩在刺你们 \\ \hline
\rsword{nofun\highlight{agihe}} & \highlight{因为}我在刺他 \\ \hline
\rsword{nonahat\highlight{agihe}} & \highlight{因为}你在打我 \\ \hline
\rsword{nahalan\highlight{agihe}} & \highlight{因为}你们将打我 \\ \hline
\rsword{nofut\highlight{agihe}} & \highlight{因为}我们在刺他 \\ \hline
\rsword{noho’in\highlight{agihe}} & \highlight{因为}我俩在打他 \\ \hline
\end{tabular}

\rsword{-agihe} 代表着\sq{因为}。

接着,我根据主语来进行分类:

\begin{tabular}[t]{l|l}
\hline
\multicolumn{2}{l}{我} \\ 
\hline
\rsword{nohobe} & 在打他 \\
\rsword{nofunagihe} & 在刺他 \\
\rsword{lenahalube} & 将打你们 \\
\\
\hline
\hline
\multicolumn{2}{l}{你} \\ 
\hline
\rsword{nofine} & 在刺他 \\
\rsword{nonahatagihe} & 在打我 \\
\\
\hline
\end{tabular}
\begin{tabular}[t]{l|l}
\hline
\multicolumn{2}{l}{我俩} \\ 
\hline
\rsword{nokoho\highlight{’i}be} & 在打你 \\
\rsword{nolenufu\highlight{’i}nagihe} & 在刺你们 \\
\rsword{lenifilu\highlight{’i}be} & 将刺你们 \\
\rsword{noho\highlight{’i}nagihe} & 在打他 \\
\hline
\hline
\multicolumn{2}{l}{你俩} \\ 
\hline
\rsword{nolifi\highlight{’i}be} & 在刺我们 \\
\rsword{nifila\highlight{’i}be} & 将刺我 \\
\rsword{lahala\highlight{’i}be} & 将打我们 \\
\hline
\end{tabular}
\begin{tabular}[t]{l|l}
\hline
\multicolumn{2}{l}{我们} \\ 
\hline
\rsword{kahalune} & 将打你 \\
\rsword{nofutagihe} & 在刺他 \\
\\
\\
\hline
\hline
\multicolumn{2}{l}{你们} \\ 
\hline
\rsword{nahalanagihe} & 将打我 \\
\\
\\
\hline
\end{tabular}

\rsword{-’i} 代表两个人。

除此之外,在比赛的时候,我没能从一张类似的表格里找出其它任何规律,
包括为什么是\sq{我俩}而不是\sq{你俩}。看这张表格的读者可能会感觉规律足够明显了,但这跟表格的结构有关,
我会在后面继续讨论这个问题。在此之前,让我们先找出表示\sq{打}和\sq{刺},以及现在时和将来时的词缀。

\begin{tabular}[t]{l|l}
\hline
\multicolumn{2}{l}{在打} \\
\hline
\rsword{\hlb{no}\hla{ho}be} & 我在打他 \\
\rsword{\hlb{no}ko\hla{ho}’ibe} & 我俩在打你 \\
\rsword{\hlb{no}na\hla{ha}tagihe} & 因为你在打我 \\
\rsword{\hlb{no}\hla{ho}’inagihe} & 因为我俩在打他 \\
\\
\hline
\hline
\multicolumn{2}{l}{将打} \\
\hline
\rsword{ka\hla{ha}\hlb{lu}ne} & 我们将打你 \\
\rsword{lena\hla{ha}\hlb{lu}be} & 我将打你们 \\
\rsword{na\hla{ha}\hlb{la}nagihe} & 因为你们将打我 \\
\rsword{la\hla{ha}\hlb{la}’ibe} & 你俩将打我们 \\
\hline
\end{tabular}
\quad
\begin{tabular}[t]{l|l}
\hline
\multicolumn{2}{l}{在刺} \\
\hline
\rsword{\hlb{no}lenu\hla{fu}’inagihe} & 因为我俩在刺你们 \\
\rsword{\hlb{no}li\hla{fi}’ibe} & 你俩在刺我们 \\
\rsword{\hlb{no}\hla{fu}nagihe} & 因为我在刺他 \\
\rsword{\hlb{no}\hla{fi}ne} & 你在刺他 \\
\rsword{\hlb{no}\hla{fu}tagihe} & 因为我们在刺他 \\
\hline
\hline
\multicolumn{2}{l}{将刺} \\
\hline
\rsword{ni\hla{fi}\hlb{la}’ibe} & 你俩将刺我 \\
\rsword{leni\hla{fi}\hlb{lu}’ibe} & 我俩将刺你们 \\
\\
\\
\hline
\end{tabular}

\rsword{no-} 和 \rsword{-lV-} 分别表示现在时与将来时;

\rsword{-hV-} 和 \rsword{-fV-} 分别表示\sq{打}与\sq{刺}。

接着我就卡住了,我把上面提到的主语表又反复抄了两遍,也按照宾语进行了排序,
但都没有取得有意义的进展。直到有一霎那,我突然想到一个问题,既然两个动词分别只有两种形式,
\rsword{-ho-}、\rsword{-ha-}、\rsword{-fi-} 和 \rsword{-fu-},
我能不能依次排序进行分析呢?

\begin{tabular}[t]{l|l}
\hline
\multicolumn{2}{l}{\rsword{-ho-}} \\
\hline
\rsword{\hlb{no}\hla{ho}be} & 我在打他 \\
\rsword{\hlb{no}ko\hla{ho}’ibe} & 我俩在打你 \\
\rsword{\hlb{no}\hla{ho}’inagihe} & 我俩在打他 \\
\\
\\
\hline
\hline
\multicolumn{2}{l}{\rsword{-fu-}} \\
\hline
\rsword{\hlb{no}lenu\hla{fu}’inagihe} & 我俩在刺你们 \\
\rsword{\hlb{no}\hla{fu}nagihe} & 我在刺他 \\
\rsword{\hlb{no}\hla{fu}tagihe} & 我们在刺他 \\
\\
\hline
\end{tabular}
\begin{tabular}[t]{l|l}
\hline
\multicolumn{2}{l}{\rsword{-ha-}} \\
\hline
\rsword{ka\hla{ha}\hlb{lu}ne} & 我们将打你 \\
\rsword{lena\hla{ha}\hlb{lu}be} & 我将打你们 \\
\rsword{\hlb{no}na\hla{ha}tagihe} & 你在打我 \\
\rsword{na\hla{ha}\hlb{la}nagihe} & 你们将打我 \\
\rsword{la\hla{ha}\hlb{la}’ibe} & 你俩将打我们 \\
\hline
\hline
\multicolumn{2}{l}{\rsword{-fi-}} \\
\hline
\rsword{leni\hla{fi}\hlb{lu}’ibe} & 我俩将刺你们 \\
\rsword{\hlb{no}li\hla{fi}’ibe} & 你俩在刺我们 \\
\rsword{\hlb{no}\hla{fi}ne} & 你在刺他 \\
\rsword{ni\hla{fi}\hlb{la}’ibe} & 你俩将刺我 \\
\hline
\end{tabular}

从表中可以轻松的看出,\rsword{-ho-} 和 \rsword{-fu-} 为一对,
\rsword{-ha-} 和 \rsword{-fi-} 为另一对。前者用于主语为第一人称、
时态为现在时的情况下,后者适于其它所有情况。

由于我把句子按照人称排序,读者们也可以发现,\rsword{-lv-} 用于第一人称,
而 \rsword{-la-} 用于第二人称。

后面还有两个词缀有待分析,就让读者们自己完成吧。

\subsection{经验教训}

这道题的平均得分极高,据我了解两支队伍也没有人卡在这里。但还是有其他一些人,
包括两三位像我这样的学习人员,都卡在这里。

问题出在对错误的变量展开研究。

我将句子根据主语进行了分类,总共有六种可能性。我将句子根据动词的变化形式进行分类,
总共只有两种可能性。数据密度增加了三倍。对于罗赛塔石碑题来说,数据密度越大,
题目难度也就越低\cite{RosettaAnalysis}。别的类型的题目也有类似的规律,比如说第二题。

前面提到,读者是有可能从我的按主语分类的句子中找出规律的。但是,我最初画的表格类似这样:

\begin{tabular}{|l|l|} \hline
主语类型 & 动词形式 \\ \hline
\multirow{3}{*}{我} &
\rsword{nohobe} \\
& \rsword{nofunagihe} \\
& \rsword{lenahalube} \\
\hline
\multirow{2}{*}{我们} &
 \rsword{kahalune} \\
& \rsword{nofutagihe} \\
\hline
\multirow{4}{*}{我俩} &
\rsword{nokoho’ibe} \\
& \rsword{nolenufu’inagihe} \\
& \rsword{lenifilu’ibe} \\
& \rsword{noho’inagihe} \\
\hline
\multirow{3}{*}{你俩} &
\rsword{nolifi’ibe} \\
& \rsword{nifila’ibe} \\
& \rsword{lahala’ibe} \\
\hline
\multirow{2}{*}{你} &
\rsword{nofine} \\
& \rsword{nonahatagihe} \\
\hline
\multirow{1}{*}{你们} &
\rsword{nahalanagihe} \\
\hline
\end{tabular}

对于主语类型,我是按照第一个该类型的句子在文本中出现的位置排序的。对于每个类型下的句子,
也是按照其在原数据集内相对顺序进行排序的。这是一种很自然的处理方法——从上往下抄数据,
但对解题无益。

与之相对应的,则是对比动词不同形式的那张表。在比赛过程中,我画的表格比这个要糟糕一些,
比如没有根据人称和时态进行排序,导致除了动词变化形式外的其它规律并没有那么明显。
可见,表的形式结构,对规律的发现有很大影响——我们的大脑处理较大数据的能力还不够强。

总结一下:

一、优先分析变化较少的变量。

二、尽可能的对数据进行有意义的排序,方便在一张表格内进行多次分析。

三、将已经发现的词缀用别的颜色标出。

当然,这里有一个问题,对于像这道题这样的数据,如何快速处理,高效的写出表格呢?
一个简单的方法,就是对其标上序号,然后先使用序号进行整理,
接着再把句子或句子的一部分抄写下来进行分析。

一些有可能成为分化条件的特征,比如代表现在时和将来时的词缀,可以提前整理出来,
抄写句子的时候,是用别的颜色将其标出。

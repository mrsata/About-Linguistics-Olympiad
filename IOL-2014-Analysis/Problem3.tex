\newcommand \tangkin [3]{\word{#1 #2 ’yn¹ #3 ngu².}}

\def \tangFa {wia¹}
\def \tangMo {ma¹}
\def \tamBr {lio²}
\def \tafBr {mu¹}
\def \tamSi {ndọn¹}
\def \tafSi {kậj¹}
\def \txCoz {źwẹj¹}
\def \tFaBr {wiej²}
\def \tMoBr {’iǝ¹}
\def \tFaSi {ny¹}
\def \tMoSi {la²}
\def \Hoo {Mbe²phon¹}% sun white
\def \Huu {Ngwi¹mbyn²}% endurance high
\def \Hooyon {Wa²nie¹}% wealth mind
\def \Huuyon {Ḳei¹źey²}% gold bear
\def \Teeyon {Sie¹tsie¹}% wisdom bright
\def \Tiiyon {Śan¹nia¹}% mountain black
\def \Tee {Lhie²nyn²}% moon red
\def \Tii {Ngon²ngwe¹}% sea blue
\def \Hoowen {Syn¹mei¹}% love eyes
\def \Huuwen {Sei¹na¹}% purity deep
\def \Teewen {Ldiu²śe¹}% beauty hot
\def \Tiiwen {Nie²tṣe¹}% pearl rabbit

\section{第三题}

这题是最受选手欢迎的一道题,依然出自戴谊凡先生之手。本题难度不大,但数据较多,看起来比较吓人。

在白高大国,即西夏,生活着两个兄弟和两个姐妹。他们每个人都有一个儿子和一个女儿。
下面用西夏语描述了各人之间的亲属关系。其中 \word{\Hoo} 是一个男子的名字。

意为\sq{父亲}和\sq{母亲}的词念第一种声调。

\centerline{\small\parbox{3.2in}{\begin{enumerate}\setlength{\itemsep}{1pt}
\item \tangkin{\Teewen}{\Tiiwen}{\tafSi}
\item \tangkin{\Hoo}{\Huu}{\tamBr}
\item \tangkin{\Huuwen}{\Hooyon}{\tamSi}
\item \tangkin{\Teewen}{\Tiiyon}{\tamSi}
\item \tangkin{\Tiiyon}{\Hoowen}{\txCoz}
\item \tangkin{\Teewen}{\Hoowen}{\txCoz}
\item \tangkin{\Teewen}{\Hooyon}{\txCoz}
\item \tangkin{\Huuyon}{\Teewen}{\txCoz}
\item \tangkin{\Teewen}{\Huuyon}{\txCoz}
\item \tangkin{\Hoo}{\Hoowen}{\tangFa}
\item \tangkin{\Hoo}{\Teeyon}{\tMoBr}
\item \tangkin{\Huuyon}{\Hooyon}{\tamBr}
\item \tangkin{\Huuyon}{\Teeyon}{\txCoz}
\item \tangkin{\Hoo}{\Huuyon}{\tFaBr}
\item \tangkin{\Hoowen}{\Tiiwen}{\txCoz}
\item \tangkin{\Tii}{\Tee}{\tafSi}
\item \tangkin{\Hoowen}{\Huuwen}{\tafSi}
\item \tangkin{\Teewen}{\Teeyon}{\tamSi}
\end{enumerate}}
\parbox{3.2in}{\begin{enumerate}\setlength{\itemsep}{1pt}\setcounter{enumi}{18}
\item \tangkin{\Hoo}{\Hooyon}{\tangFa}
\item \tangkin{\Teewen}{\Huuwen}{\txCoz}
\item \tangkin{\Hooyon}{\Huuwen}{\tafBr}
\item \tangkin{\Tiiyon}{\Hooyon}{\txCoz}
\item \tangkin{\Tii}{\Teewen}{\tMoSi}
\item \tangkin{\Hoo}{\Tee}{\tafBr}
\item \tangkin{\Tee}{\Teewen}{\tangMo}
\item \tangkin{\Hoo}{\Tiiwen}{\tMoBr}
\item \tangkin{\Hoo}{\Tii}{\tafBr}
\item \tangkin{\Tiiyon}{\Teeyon}{\tamBr}
\item \tangkin{\Hoo}{\Teewen}{\tMoBr}
\item \tangkin{\Hoo}{\Huuwen}{\tFaBr}
\item \tangkin{\Tiiyon}{\Tiiwen}{\tafBr}
\item \tangkin{\Hoo}{\Tiiyon}{\tMoBr}
\item \tangkin{\Tii}{\Huuyon}{\tFaSi}
\item \tangkin{\Hoowen}{\Tiiyon}{\txCoz}
\item \tangkin{\Tee}{\Teeyon}{\tangMo}
\item \tangkin{\Tiiwen}{\Teeyon}{\underline{\qquad}}
\end{enumerate}}}

\subsection{推理过程}

首先,我们可以清楚的发现,所有的句子都呈现以下形式:

\word{A B ’yn¹ X ngu².}

因此,我们大致可以猜测,每一句的意思 “A 是 B 的 X” 或 “B 是 A 的 X”。
需要注意的是,这两种表述并不等价——前者意味着 X 对 A 有限制,后者意味着 X 对 B 有限制。
具体是哪种情况,还有待进一步分析。

简单的数数便可以发现,牵涉到 \word{\Hoo} 的共有十一个关系,而题目中总共也就只有十二个人,
四个父辈及他们每个人的两个子女。换句话说,\word{\Hoo} 和所有人的关系都已经被描述了。
自然的,我们开始对牵涉到他的句子进行分析。

\begin{tabular}{l|l}
\hline
\word{\Hoo} 与谁 & 关系 \\
\hline
\word{\Huu} & \word{\tamBr} \\
\word{\Hoowen} & \word{\tangFa} \\
\word{\Hooyon} & \word{\tangFa} \\
\word{\Tee} & \word{\tafBr} \\
\word{\Tii} & \word{\tafBr} \\
\word{\Huuyon} & \word{\tFaBr} \\
\word{\Huuwen} & \word{\tFaBr} \\
\word{\Teeyon} & \word{\tMoBr} \\
\word{\Tiiwen} & \word{\tMoBr} \\
\word{\Teewen} & \word{\tMoBr} \\
\word{\Tiiyon} & \word{\tMoBr} \\
\hline
\end{tabular}

鉴于有四个人和 \word{\Hoo} 的关系是用同一个词来表示,可以确定 \word{\Hoo} 是父辈,
这四人为子辈;只出现了一次的关系 \word{\tamBr} 必然表示同辈关系,且两者同性。
剩下三种关系,其中两种表示异辈关系,还有一种则为同辈关系。

还有两句话也出现了 \word{\tamBr}:

\tangkin{\Huuyon}{\Hooyon}{\tamBr}

\tangkin{\Tiiyon}{\Teeyon}{\tamBr}

可以得出以下几点:

一、\word{\Huuyon} 和 \word{\Hooyon} 同辈;\word{\Tiiyon} 和 \word{\Teeyon} 也同辈。
后两者和 \word{\Hoo} 的关系是 \word{\tMoBr},因此后两者必为子辈;前两者是否为子辈尚待观察。

二、\word{\Huuyon} 和 \word{\Hooyon} 同性;\word{\Tiiyon} 和 \word{\Teeyon} 也同性。
四者是否同性尚待观察,比如说英语中的 “cousin”,就没有性别要求。

三、\word{\Huu} 和 \word{\Hoo} 的关系是亲兄弟或姐妹,但每个父辈都只有一男一女两个后代,
因此,汉语中\sq{表哥}和\sq{亲哥}这样的亲属关系,在西夏语中不作区分。

下面来分析另外一个反复在材料中出现,但尚未遇到过的新词,\word{\txCoz}:

\begin{enumerate}
\item \tangkin{\Tiiyon}{\Hoowen}{\txCoz}
\item \tangkin{\Teewen}{\Hoowen}{\txCoz}
\item \tangkin{\Teewen}{\Hooyon}{\txCoz}
\item \tangkin{\Huuyon}{\Teewen}{\txCoz}
\item \tangkin{\Teewen}{\Huuyon}{\txCoz}
\item \tangkin{\Huuyon}{\Teeyon}{\txCoz}
\item \tangkin{\Hoowen}{\Tiiwen}{\txCoz}
\item \tangkin{\Teewen}{\Huuwen}{\txCoz}
\item \tangkin{\Tiiyon}{\Hooyon}{\txCoz}
\item \tangkin{\Hoowen}{\Tiiyon}{\txCoz}
\end{enumerate}

这是表示同辈还是异辈关系呢?\word{\Huuyon} 和 \word{\Teewen} 是 \word{\txCoz},
\word{\Teewen} 和 \word{\Hooyon} 也是 \word{\txCoz},
加上 \word{\Huuyon} 和 \word{\Hooyon} 同辈,可见 \word{\txCoz} 表示同辈关系。
简单的分析可以发现,这上面的十个句子提到的所有人均为同辈,正好八个人。

回到对 \word{\Hoo} 之前的分析,我们可以确认,父辈的人有:
\word{\Hoo}、\word{\Huu}、\word{\Tee} 和 \word{\Tii}。

\word{\Hoo} 与所有其他人的关系均以呈现,意味着必然会提到父母关系。
因为 \word{\Hoo} 与四人的关系均为 \word{\tMoBr},但 \word{\Hoo} 只有两个子女,
所以 \word{\tMoBr} 不可能表示父母关系。对于剩下两个牵涉到 \word{\Hoo} 的异辈关系,
只有 \word{\tangFa} 为第一种声调,因此其表示父母关系,
\word{\Hoowen} 和 \word{\Hooyon} 为 \word{\Hoo} 的子女。

我们还有很多数据没有分析,如果对子辈先进行分类,比如确定谁和谁是亲生兄弟姐妹,
谁和谁是表兄弟姐妹,想来对做题很有帮助。一个典型的做法,便是构建一张图,点集为八个子辈,
边集为所有描述子辈之间关系的集合:

\SetVertexNormal[
  Shape=circle,
  FillColor=white,
  LineWidth=1pt]
\tikzset{VertexStyle/.append style={font=\bfseries\ipafont}}
\SetUpEdge[
  lw=1pt,
  color=black,
  labelcolor=white,
  labelstyle={font=\bfseries\ipafont}]
\begin{center}
\begin{tikzpicture}
  \Vertex[x=3, y=9]{Sm}
  \Vertex[x=0, y=6]{Wn}
  \Vertex[x=0, y=3]{Sn}
  \Vertex[x=3, y=0]{Kź}
  \Vertex[x=7, y=9]{Lś}
  \Vertex[x=10, y=6]{St}
  \Vertex[x=10, y=3]{Nt}
  \Vertex[x=7, y=0]{Śn}
  \tikzset{EdgeStyle/.style={->}}
  \Edge[label=\tafSi](Sm)(Sn)
  \Edge[label=\txCoz](Sm)(Nt)
  \tikzset{EdgeStyle/.style={->,relative=false,in=180,out=180}}
  \Edge[label=\tafBr](Wn)(Sn)
  \tikzset{EdgeStyle/.style={->,relative=false,in=260,out=100}}
  \Edge[label=\tamSi](Sn)(Wn)
  \tikzset{EdgeStyle/.style={->}}
  \Edge[label=\txCoz](Kź)(St)
  \Edge[label=\tamBr](Kź)(Wn)
  \Edge[label=\txCoz](Lś)(Sm)
  \tikzset{EdgeStyle/.style={->,relative=false,in=90,out=120}}
  \Edge[label=\txCoz](Lś)(Wn)
  \tikzset{EdgeStyle/.style={->}}
  \Edge[label=\txCoz](Lś)(Sn)
  \Edge[label=\tamSi](Lś)(St)
  \tikzset{EdgeStyle/.style={->,relative=false,in=87,out=273}}
  \Edge[label=\tamSi](Lś)(Śn)
  \tikzset{EdgeStyle/.style={->}}
  \Edge[label=\tafSi](Lś)(Nt)
  \Edge[label=\txCoz](Śn)(Wn)
  \Edge[label=\tamBr](Śn)(St)
  \Edge[label=\tafBr](Śn)(Nt)
  \tikzset{EdgeStyle/.style={<->}}
  \Edge[label=\txCoz](Sm)(Śn)
  \Edge[label=\txCoz](Kź)(Lś)
\end{tikzpicture}
\end{center}

出于美观考虑,上图中所有的名字顶点均使用缩写,两个字母是名字的两个音节的首辅音字母。

八个点被分为两组,第一组为 \word{\Hoowen}、\word{\Hooyon}、\word{\Huuwen} 和 \word{\Huuyon},
第二组为 \word{\Teewen}、\word{\Teeyon}、\word{\Tiiwen} 和 \word{\Tiiyon}。
\word{\txCoz} 关系只出现在组间,其余同辈关系均出现在组内。

小组按什么规则进行划分的呢?已知的唯一一对亲兄弟姐妹,\word{\Hoowen} 和 \word{\Hooyon},出现在同一组。
可以判断出,这一组剩下的两人,也是亲兄弟姐妹,而且同性两人的子女会被分为一组——
如果是异性的话,无法确定分组:为什么一个男人的子女,要和他的姐姐(妹妹)的子女分为一组,
而不是和他的妹妹(姐姐)的子女分为一组?

根据前面关于父辈的讨论,\word{\Tee} 和 \word{\Tii} 为父辈的那对兄弟或姐妹。
牵涉到两者的表示异辈关系的词分别有 \word{\tFaSi}、\word{\tMoSi} 和 \word{\tangMo}。
虽然 \word{\tFaSi} 也为第一种声调,但与 \word{\Tii} 有此关系的 \word{\Huuyon},
却是和 \word{\Hoowen} 为一组,后者的父亲或母亲为 \word{\Hoo},与 \word{\Tii} 异性。
因此,\word{\Tii} 不可能是 \word{\Huuyon} 的父亲或母亲。因此,
\word{\tangMo} 是另外一个表示亲生关系的词。

那么,\word{\tangMo} 和 \word{\tangFa},谁代表父亲,谁代表母亲呢?我们可以发现,
\word{\tangMo} 的发音,类似各种语言中的母亲的发音,因此,\word{\tangMo} 代表母亲,
\word{\tangFa} 代表父亲。每组剩下的一对男女的父母也随之确定。

之前我们无法分辨 “\word{A B ’yn¹ X ngu².}” 表示的是 “A 是 B 的 X” 还是 “B 是 A 的 X”。
但在描述父亲关系的词中,父亲出现在 A,因此,我们可以断定这个句式表示的是 “A 是 B 的 X”。

现在,我们已经知道的有:

一、父辈四个人的性别。

二、子辈八个人的父母。

三、子辈中,\word{\Huuyon} 和 \word{\Hooyon} 同性;\word{\Tiiyon} 和 \word{\Teeyon} 同性。

稍作观察即可发现,描述同一组内亲属关系的共有四个词,分别是 \word{\tamBr}、\word{\tamSi}、
\word{\tafBr} 和 \word{\tafSi}。考虑到在 “A 是 B 的 X” 这样的表述中,
根据 A 和 B 性别不同,总共有四种可能,可以确定,这四个词对 A 与 B 的性别均有要求。
由于 \word{\tamBr} 用于两位男性之间,类似汉语中的关系\sq{兄弟},我们可以确定子辈八人的性别,并推出下表:

\begin{tabular}{c|c|l}
\hline
A & B & X \\
\hline
男 & 男 & \word{\tamBr} \\ \hline
女 & 男 & \word{\tamSi} \\ \hline
男 & 女 & \word{\tafBr} \\ \hline
女 & 男 & \word{\tafSi} \\ \hline
\end{tabular}

需要注意,本推理过于严谨,在实际做题的时候,没有必要。比如说,\word{\txCoz} 出现了十次,
又和 \word{\Hoo} 无关,便可以肯定这是描述子辈之间的一种极为松散的关系,且双方父母不同,
亲属关系较远,因为只有这样,才没必要分的十分仔细。

这是一道大约能在一个半小时完成的题目,有趣又不难。

\subsection{拓展阅读}

戴谊凡先生在闭幕式的说明中提到,西夏语的亲属系统很像塞内卡语的亲属系统。
下面给出了塞内卡语亲属系统的一组数据\cite{Seneca}。塞内卡语属于易洛魁语系,
在纽约州,还剩一百人使用这门语言。

表格的举例列使用缩写标记表达亲属关系:

F = Father, M = Mother, S = Sister, B = Brother, s = son, d = daughter.

标记应当从左往右当作一串亲属关系描述来理解,比如说 “FMSs”,就表示父亲的母亲的姐姐的儿子。

\sq{堂表}是指堂或表兄弟姐妹。

\begin{center}
\begin{tabular}{|c|c|p{10cm}|}
\hline
塞内卡语 & 汉语翻译 & \parbox[t]{10cm}{\centering{举例}} \\ \hline
\word{haʔnih} & 我的父亲 & F, FB, FMSs, FFBs, FMBs, FFSs, FFFBss \\ \hline
\word{noʔyẽh} & 我的母亲 & M, MS, MMSd, MFBd, MMBd, MFSd, MMMSdd \\ \hline
\word{hakhno ́ʔsẽh} & 我的叔叔 & MB, MMSs, MFBs, MMBs, MFSs, MMMSds \\ \hline
\word{ake:hak} & 我的阿姨 & FS, FMSd, FFBd, FMBd, FFSd, FFFBsd \\ \hline
\word{hatsiʔ} & 我的哥哥 & 
\parbox[t]{10cm}{B, MSs, FBs, MMSds, FFBss, MFBds, FMSss, MMBds\\比自己大} \\ \hline
\word{heʔkẽ:ʔ} & 我的弟弟 & 同上,但比自己小 \\ \hline
\word{ahtsiʔ} & 我的姐姐 & 
\parbox[t]{10cm}{S, MSd, FBd, MMSdd, FFBsd, MFBdd, FMSsd, MMBdd\\比自己大} \\ \hline
\word{kheʔkẽ:ʔ} & 我的妹妹 & 同上,但比自己小 \\ \hline
\word{akyã́:ʔse:ʔ} & 我的堂表 & 
\parbox[t]{10cm}
{MBs, FSs, MMSss, FFBds, MFBss, FMSds, MMBss\\
也可以是:\\
MBd, FSd, MMSsd, FFBdd, MFBsd, FMSdd, MMBsd} \\ \hline
\word{he:hawak} & 我的儿子 & 
\parbox[t]{10cm}{自己为男性:\\s, Bs, MSss, FBss, MBss, FSss, MMSdss} \\ & &
\parbox[t]{10cm}{自己为女性:\\s, Ss, MSds, FBds, MBds, FSds, MMSdds} \\ \hline
\word{khe:hawak} & 我的女儿 & 
\parbox[t]{10cm}{自己为男性:\\d, Bd, MSsd, FBsd, MBsd, FSsd, MMSdsd} \\ & &
\parbox[t]{10cm}{自己为女性:\\d, Sd, MSdd, FBdd, MBdd, FSdd, MMSddd} \\ \hline
\word{heyẽ́:wõ:tĕʔ} & 我的侄子 & 
\parbox[t]{10cm}{自己为男性:\\Ss, MSds, FBds, MBds, FSds, MMSdds} \\
\word{hehsṍʔneh} & &
\parbox[t]{10cm}{自己为女性:\\Bs, MSss, FBss, MBss, FSss, MMSdss} \\ \hline
\word{kheyẽ́:wõ:tĕʔ} & 我的侄女 & 
\parbox[t]{10cm}{自己为男性:\\Sd, MSdd, FBdd, MBdd, FSdd, MMSddd} \\
\word{khehsṍʔneh} & &
\parbox[t]{10cm}{自己为女性:\\Bd, MSsd, FBsd, MBsd, FSsd, MMSdsd} \\ \hline
\end{tabular}
\end{center}

很容易看出,塞内卡语的亲属关系和西夏语很接近。读者可以尝试分析并准确的定义每一个塞内卡语亲属关系,
找出其内在规律。
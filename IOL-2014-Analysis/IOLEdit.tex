\section{试题的制作与翻译}

先简单八卦一下 IOL 试题的制作吧。

第一届 IOL 试题使用 \LaTeX 书写,像数据这样的在不同语言版本中保持一致的文本被定义为宏,
接着在各个语言的版本中被引用。第二届至第五届的 IOL 试题均适用 Word 编写。从第六届开始,
由于 IOL 工作语言日益增多,维护多份 Word 文稿的成本日益增加,
组委会再一次选择 \LaTeX\cite{IOLEdit}。

如何保证不同语言版本的试题绝对公平呢?组委会想出了一个办法:
把各个语言的翻译切割成较小单位,如句子或者词组,制成一张二维表,进行比较。
一些比较常见的表述也会被抽取出来,比如\sq{下面是……及其汉语翻译},
及其对应的英文翻译 “Here are ... and their English translations.”。
这些基本单位和常用表述被定义为宏,在主 LaTeX 文稿里\sq{组装}起来。

感谢伟大的 \westname{Donald Knuth},\TeX 有了很多很棒的特性,
比如说,\TeX 的宏系统使用 dynamic scope,这样在调用的宏里可以访问被调用方声明的宏;
此外,为了节约空间,\TeX 使用八位来表示一个字符,而我们常见的其它文本编码,
需要花三四倍的空间来表示汉字。

为什么要尝试翻译 IOL 试题呢?IOL 2013 结束后,我在学校开始宣传 IOL,
突然发现,一整张纸的英文竟然吓跑了许多同学,翻译势在必行;
此外,了解到 \LaTeX 对 CJK (Chinese, Japanese \& Korean) 的糟糕支持,
而明年组委会将使用中文赛题,我自认为有义务帮助组委会扫除障碍。
于是,我便联系了戴谊凡 (\westname{Ivan Derzhanski}) 先生,开始了试题翻译计划。
去年年底,IOL 2012 试题的第一个版本完成了,但是出于两个原因,
IOL 试题的模块化结构(即一个文件存储语言相关文本,另外一个主文件构建文档)被破坏殆尽。
其一,如前所述,主文件使用了预定义的宏构建而成,这些宏调用之间预留了空格,在汉语中是不恰当的,
需要删去。其次,主文件硬编码了标点符号,而按照规范,汉语文档应当使用全角标点。

二月初,曹起曈加入了我,并说服我使用半角标点。与此同时,我写了一个程序,
自动删除主文件中的多余空格,IOL 翻译计划的技术障碍基本扫除。然而由于一些我个人的原因,
IOL 2008 拖到今年八月份,在刘晗的帮助下才完成。至于 IOL 2003-2007 由于没有原始文档,
翻译暂时搁置,不过目前陈润已经完成了 IOL 2003 个人赛赛题的文本的翻译。

不过这篇文档是关于 IOL 2014 的,还是把目光转向 IOL 2014 的翻译吧。
IOL 2014 的翻译最初是由组委会负责的,按照前述的制表工作,理论上绝对公平,
后期王伟帮助组委会继续翻译,让表达更加本地化。由于没有参加 IOL 2014(其实是落选),
我在赛前也帮助戴先生检查排版错误,并从选手的视角找寻可能导致不公平的地方。

把校对中遇到和赛后发现的错误总结一下,这三个比较典型:

\begin{enumerate}
\item 个人赛第四题的初稿里使用了\sq{女人长得像这个姑娘吗?},但同样的句子在英文翻译中体现了过去时。

\item 个人赛第五题,动词\sq{放}存在歧义,是\sq{放置}还是\sq{放过}。

\item 团体赛,\bookref{世界人权宣言},将英文中的 “national origin” 和 “nationality” 均翻译成国籍。
\end{enumerate}

这三个小问题,给了我们三个教训。

其一,翻译人员并没有参与到试题的讨论中去。恩盖尼语实际上不区分过去时和现在时,作者为了简化题目,
才让英文翻译体现出过去时。而在西班牙语翻译则体现了两种不同时态,但组委会认为这两种时态共性明显,
不妨碍选手将其归为一类。

其二,翻译的歧义需要仔细检查,尤其是语义题的翻译。

其三,对于一切文本都尽量使用表格来检查。在闭幕式后我询问了戴先生,
对于\bookref{世界人权宣言},他们直接拷贝了联合国的官方翻译,并未进行任何检查。

顺便说一句,据我估算,组委会在开幕式前两天,即十九日,才完成中文试题,
我相信他们一定遇到了时间不足的问题。立志进入 Problem Committee 的同学们,
可以现在就去写\href{mailto:pc-chair@ioling.org}{邮件}。

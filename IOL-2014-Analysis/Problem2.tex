\newcommand \allshade {\cellcolor[gray]{.8}}

\section{第二题}

\begin{tabular}{|l|l|l|l|}\hline
单数 & 双数 & 复数 & \\ \hline
\bipa{adɔ} & \bipa{a} & \bipa{a} & 树 \\ \hline
\bipa{matʰɔnsjan} & \bipa{matʰɔnsjan} & \bipa{matʰɔnsjadɔ} & 小女孩 \\ \hline
\bipa{k’ɔ} & \bipa{k’ɔ} & \bipa{k’ɔgɔ} & 刀 \\ \hline
\bipa{tʰot’olagɔ} & \bipa{tʰot’ola} & \bipa{tʰot’olagɔ} & 橙子 \\ \hline
\bipa{aufi} & \allshade & \bipa{aufigɔ} & 鱼 \\ \hline
\bipa{pʰjaboadɔ} & \allshade & \bipa{pʰjaboa} & 路灯 \\ \hline
\bipa{matʰɔn} & \allshade & \bipa{matʰɔdɔ} & 姑娘 \\ \hline
\bipa{k’ɔnbohodɔ } & \allshade & \bipa{k’ɔnbohon} & 帽子 \\ \hline
\bipa{t’ɔ} & \allshade & \bipa{t’ɔgɔ} & 勺子 \\ \hline
\allshade & \allshade & \bipa{e} & 面包 \\ \hline
\bipa{alɔsɔhjegɔ} & ? & \bipa{alɔsɔhjegɔ} & 李 \\ \hline
? & \bipa{tsegun} & \bipa{tsegudɔ} & 狗 \\ \hline
\bipa{alɔguk’ogɔ} & \bipa{alɔguk’o} & ? & 柠檬 \\ \hline
? & \bipa{k’apʰtʰɔ} & \bipa{k’apʰtʰɔgɔ} & 老汉 \\ \hline
\bipa{kʰɔdɔ} & \bipa{kʰɔ} & ? & 被子 \\ \hline
\bipa{k’ɔdɔ} & ? & \bipa{k’ɔdɔ} & 西红柿 \\ \hline
? & \bipa{alɔ} & ? & 苹果 \\ \hline
? & \bipa{pʰɔ} & ? & 野牛 \\ \hline
? & ? & \bipa{sadɔ} & 儿童 \\ \hline
\bipa{ɔlsun} & ? & ? & 梳子 \\ \hline
? & \bipa{pitso} & ? & 叉子 \\ \hline
? & \bipa{tʰɔpʰpaa} & ? & 椅子 \\ \hline
\end{tabular}

上表为一些基奥瓦语名词的单双复数形式及其汉语翻译。表中名词均有三种形式,
但没有全部列出。基奥瓦语属于基奥瓦-塔诺安语系,濒临灭亡,只剩美国奥克拉荷马州的几百人仍在使用。

这题的作者是 \westname{Aleksejs Peguševs},帅哥,我不熟悉。
本题似乎是仅次于最后一题的一道难题,但我感觉还好。

一般来说,这样的表格题都不需要关心翻译,但本题是个例外。作者在闭幕式的演讲中也提到,
常见的错误之一便是只考虑了音韵变化,而没有考虑语义。比如说,迅速放弃这道题的曹起曈。

\subsection{推理过程}

首先,对发生的形态变化进行分类,很容易发现,这道题的变化无非就是添加一个后缀,
后缀有两种形式,“\word{gɔ}” 和 “\word{dɔ}”。变化可能发生在单数形式,也可能发生在复数形式,
也可能同时发生在单数和复数形式里。

语言的形式承载着信息。这样的形态变化,意味着什么呢?

某些时候,基奥瓦语母语者跟英语母语者一样,区分单复数。

某些时候,基奥瓦语母语者认为,一件东西和两件东西没有区别,只有三件及以上,才需要特殊标注。

某些时候,基奥瓦语母语者认为,两件东西需要特殊标注,一件或者三件及以上,没有区别。

或许他们会在名词的前面加上数词,但我想,名词的形态变化,必然会反映了基奥瓦语母语者大脑里的某些想法。

你觉得最初说基奥瓦语的那个部落的人,会因为一个单词以 “\word{-n}” 结尾,就改变对其的看法吗?

这道题的规律与名词的语义有关,便是一个很自然的推论。那就分个类吧:

\begin{tabular}[t]{l|c|c}
\hline
\multicolumn{3}{l}{单数形式变化} \\
\hline
基态 & 变化 & 翻译 \\ \hline
\bipa{a} & -\bipa{dɔ} & 树 \\ \hline
\bipa{pʰjaboa} & -\bipa{dɔ} & 路灯 \\ \hline
\bipa{kʰɔ} & -\bipa{dɔ} & 被子 \\ \hline
\bipa{k’ɔnboho\highlight{n}} & -\bipa{dɔ} & 帽子 \\ \hline
\bipa{e} & \allshade & 面包 \\ \hline
\end{tabular}
\begin{tabular}[t]{l|c|c}
\hline
\multicolumn{3}{l}{复数形式变化} \\
\hline
基态 & 变化 & 翻译 \\ \hline
\bipa{k’ɔ} & -\bipa{gɔ} & 刀 \\ \hline
\bipa{aufi} & -\bipa{gɔ} & 鱼 \\ \hline
\bipa{t’ɔ} & -\bipa{gɔ} & 勺子 \\ \hline
\bipa{matʰɔ\highlight{n}} & -\bipa{dɔ} & 姑娘 \\ \hline
\bipa{matʰɔnsja\highlight{n}} & -\bipa{dɔ} & 小女孩 \\ \hline
\bipa{ɔlsun} & ? & 梳子 \\ \hline
\end{tabular}
\begin{tabular}[t]{l|c|c}
\hline
\multicolumn{3}{l}{单复数形式均变化} \\
\hline
基态 & 变化 & 翻译 \\ \hline
\bipa{tʰot’ola} & -\bipa{gɔ} & 橙子 \\ \hline
\bipa{alɔsɔhjegɔ} & -\bipa{gɔ} & 李子 \\ \hline
\bipa{alɔguk’ogɔ} & -\bipa{gɔ} & 柠檬 \\ \hline
? & -\bipa{dɔ} & 西红柿 \\ \hline
\end{tabular}

红色部分表示发生形态变化后该部分被移除。

需要注意的是,即使有些行并没有给出三种形式,我们也可以根据已经给出的形式,对其进行分类。
比如说,第三人称没有发生形态变化,则必然可以归于第一类。

单复数形式均变化的,可以发现都是水果;只有复数形式发生变化的,可能是动物、人或者小工具;
只有单数形式变化的,相对比较混乱,答案中给出的是\sq{所有其它东西}。我觉得这是可以猜出来的,
比如说\sq{路灯}或者\sq{帽子},很有可能在原来这个语言中不存在,后来和欧洲侵略者交流之后,
才传进这门语言中。当然,这种看法未必正确。

再说音韵变化。可以清楚的发现,如果单词以 “\word{-n}” 结尾,“\word{-n}” 总是被去掉,
或者说变成 “\word{-d}”,再加上 “\word{-ɔ}”。虽然没有直接给出在单复数形式均变化时,
以 “\word{-n}” 结尾的单词的变化形式,但我相信这里会有同样的规则。

然而,虽然大部分情况下元音结尾的单词都直接获得词缀 “\word{-gɔ}”,但是如果单词属于第一分类,
只有单数形式变化,即使单词以元音结尾,也会获得词缀 “\word{-dɔ}”。这条规律也很容易观察。

总的来说,这道题还是极为简单的。

\subsection{经验教训}

通过对题目数据进行分析,我们可以大致了解有哪些信息需要通过对话来提供,
有哪些信息则可以被交谈双方推导出来。比如说,如果某种语言的名词没有单复数的形态变化,
那么动词上某个一直无法理解的词缀很有可能就标记着主语或者宾语的单复数。
反之,如果我们确定某一个重要的信息没有被表达出来,比如本题中单双复数的信息,
它很有可能已经附加在另外一个信息上,没有被我们发现。

另外一个教训就是,小题里的数据也要适当分析。尤其是难题,直接给出的数据可能不充足,如本题;
即使充足,借助小题里的数据,也可以降低分析难度,或者验证自己的猜想。举个例子,IOL 2013 的第四题,
动词的一个词缀似乎标记着名词的单复数,通过小题里的数据我们可以发现,这个猜想不满足全部数据,需要修正或抛弃。
